\documentclass{ximera}
      
\title{Problem 5}
      
\begin{document}
      
\begin{abstract}
      
In this activity, we are given an example problem and we try to identify what we know, what we're trying to do, and how to approach it.
      
\end{abstract}
      
\maketitle
      
 
Rats exposed to imaginarium have a 30\% chance of developing tumors, and those who don't have a 5\% chance of developing tumors.  If 80\% of the rats are exposed to imaginarium, how many rats will develop tumors?  Let $T$ denote ``tumors" and $I$ denote ``exposed to imaginarium".
 
 \begin{enumerate}
\item What do we know?

\begin{selectAll}
\choice{$P(T)$}
\choice[correct]{$P(I)$}
\choice{$P(T\ \text{and}\ I)$}
\choice{$P(T\ \text{or}\ I)$}
\choice[correct]{$P(T|I$)}
\choice[correct]{$P(T|I^c$)}
\choice{$P(I|T)$}
\choice{$P(I|T^c$)}

\end{selectAll}

\item What are we trying to find?

\begin{multipleChoice}
\choice[correct]{$P(T)$}
\choice{$P(I)$}
\choice{$P(T^c)$}
\choice{$P(I^c)$}
\choice{$P(T\ \text{and}\ I)$}
\choice{$P(T^c\ \text{and}\ I)$}
\choice{$P(I\ \text{and}\ T^c)$}
\choice{$P(I^c\ \text{and}\ T^c)$}

\choice{$P(T\ \text{or}\ I)$}
\choice{$P(T^c\ \text{or}\ I)$}
\choice{$P(T\ \text{or}\ I^c)$}
\choice{$P(T^c\ \text{or}\ I^c)$}

\choice{$P(T|I)$}
\choice{$P(T^c|I)$}
\choice{$P(T|I^c)$}
\choice{$P(T^c|I^c)$}

\choice{$P(I|T)$}
\choice{$P(I^c|T)$}
\choice{$P(I|T^c)$}
\choice{$P(I^c|T^c)$}

\end{multipleChoice}

\item The most relevant formula is:

\begin{multipleChoice}
\choice{$P(A\ \text{or}\ B)=P(A)+P(B)-P(A\ \text{and}\ B)$}
\choice{$P(A|B)=\frac{P(A\ \text{and}\ B)}{P(B)}$}
\choice[correct]{$P(B)=P(B|A)P(A)+P(B|A^c)(1-P(A))$}
\end{multipleChoice}

\item What is the actual answer?  The probability is (round to 4 decimal places if neccesary): $\answer{0.26}$

\item[Bonus:] How many rats were either exposed to imaginarium or developed tumors: $\answer{0.82}$.

\end{enumerate}

 
 
 
 
      






\end{document}
