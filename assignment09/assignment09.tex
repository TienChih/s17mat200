\documentclass{ximera}
      
\title{Probability Distributions}
      
\begin{document}
      
\begin{abstract}
      
In this activity, we examine notions of probability distributions.

\url{https://cnx.org/contents/MBiUQmmY@18.54:J3nUH8pG@9/Mean-or-Expected-Value-and-Sta}

      
\end{abstract}
      
\maketitle
 

\begin{problem}
Consider the following table:

$$
\begin{array}{c|cccc}
X&-1&2&3&4\\
\hline
P(X)&0.1&0.5 &0.2&0.2
\end{array}
$$
Find $E(X)=\answer{2.3}$ and $\sigma=\answer{1.3454}$ if the above table represents a probability distribution.  If it doesn't, write NA for both answers.  Round to 4 decimal places if necessary

\end{problem}



\begin{problem}
Consider the following table:

$$
\begin{array}{c|cccc}
X&1&2&3&4\\
\hline
P(X)&0.15&0.32 &0.28&0.15
\end{array}
$$
Find $E(X)=\answer{NA}$ and $\sigma=\answer{NA}$ if the above table represents a probability distribution.  If it doesn't, write NA for both answers.  Round to 4 decimal places if necessary

\end{problem}

\begin{problem}
Consider the following table:

$$
\begin{array}{c|cccc}
X&-1&-2&-3&-4\\
\hline
P(X)&0.15&0.32 &0.28&0.25
\end{array}
$$
Find $E(X)=\answer{-2.63}$ and $\sigma=\answer{1.0164}$ if the above table represents a probability distribution.  If it doesn't, write NA for both answers.  Round to 4 decimal places if necessary

\end{problem}

\begin{problem}
Suppose that a game of chance works thusly:  You pay \$10 and roll a dice.  If you roll a multiple of 2, you get \$10, and if you get a multiple of 3, you get \$12.  What are your expected net earnings from playing this game? \$$\answer{-1}$
\end{problem}


\begin{problem}
Suppose that out of 8 cupcakes, 3 are poisoned.  Suppose you eat 2 of them. Let $X$ denote the number of poisoned cupcakes that you eat.  Find a probability distribution table for $X$.

$$
\begin{array}{c|ccc}
X&0&1&2\\
\hline
P(X)&\answer{10/28}&\answer{15/28} &\answer{3/28}
\end{array}
$$
What is the average number of cupcakes expected to be eaten? $\answer{3/4}$.

What is $\sigma=\answer{0.6339}$ (round to 4 decimal places if necessary).

\end{problem}





\end{document}
